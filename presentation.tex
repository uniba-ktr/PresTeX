%===============================================================================
% Zweck:    KTR-Präsentation-Vorlage
% Erstellt: 15.04.2013
% Update:   04.07.2016
% Autor:    M.G.
%===============================================================================
\RequirePackage[hyphens]{url}
\newcommand\ratio{169}
\documentclass[10pt,aspectratio=\ratio,
%draft,
%handout,
compress
]{beamer}

\newcommand\meta{./meta}
\input{\meta/config/commands}


\def\signed #1{{\leavevmode\unskip\nobreak\hfil\penalty50\hskip2em
  \hbox{}\nobreak\hfil(#1)%
  \parfillskip=0pt \finalhyphendemerits=0 \endgraf}}

\newsavebox\mybox
\newenvironment{aquote}[1]
  {\savebox\mybox{#1}\begin{fancyquotes}}
  {\signed{\usebox\mybox}\end{fancyquotes}}


\input{\meta/config/hyphenation}

\setbeamertemplate{caption}[numbered]
%\numberwithin{figure}{section}

\begin{document}
  %===============================================================================
  % Zum Kompilieren latexmk ausführen.
  %	Konfiguration in texmaker: Options -> Configure Texmaker -> Quick Build -> Select Latexmk + ViewPD
  %	Entsprechende Informationen in den config/metainfo verändern
  % Zur Auswahl der Sprache im folgenden Befehl
  % ngerman für deutsch eintragen, english für Englisch.
  %===============================================================================
\selectlanguage{english}
\ifnum\ratio<169
\frame{\titlepage}
\else
\frame[plain]{\titlepage}
\fi

%\AtBeginSection[]
%{
%  \frame<handout:0>
%  {
%    \frametitle{Outline}
%    \tableofcontents[currentsection,hideallsubsections]
%  }
%}

\AtBeginSubsection[]
{
  \frame<handout:0>
  {
    \frametitle{Outline}
    \tableofcontents[sectionstyle=show/hide,subsectionstyle=show/shaded/hide,subsubsectionstyle=hide]
  }
}

\AtBeginSubsubsection[]
{
  \frame<handout:0>
  {
    \frametitle{Outline}
    \tableofcontents[sectionstyle=show/hide,subsectionstyle=show/shaded/hide,subsubsectionstyle=show/shaded/hide]
  }
}

\newcommand<>{\highlighton}[1]{%
  \alt#2{\structure{#1}}{{#1}}
}

\newcommand{\icon}[1]{\pgfimage[height=1em]{#1}}

\section*{}
\phantomsection
\begin{frame}{Content}
\tableofcontents
\end{frame}
%%%%%%%%%%%%%%%%%%%%%%%%%%%%%%%%%%%%%%%%%
%%%%%%%%%% Content starts here %%%%%%%%%%
%%%%%%%%%%%%%%%%%%%%%%%%%%%%%%%%%%%%%%%%%

\section{Logo}
\begin{frame}{Logo}
\framesubtitle{In Blau}
%#1 Breite
%#2 Datei (liegt im image Verzeichnis)
%#3 Beschriftung
%#4 Label fuer Referenzierung
\image{.25\textwidth}{\meta/config/images/logo.png}{Uni-Logo}{img:logo}
\end{frame}

\section{Git Version}
\begin{frame}{Git Version}\framesubtitle{\gitAbbrevHash}
\begin{itemize}
\item \#: \gitAbbrevHash
\item @: \gitAuthorIsoDate
\item \gitReferences
\end{itemize}
\textbf{Setup}\\
Follow this link to \href{https://www.ctan.org/tex-archive/macros/latex/contrib/gitinfo2}{Gitinfo 2}\\
\textbf{git hooks}\\
  To fill watermark at buttom, deploy gitinfo2-hook.txt to githooks: (copy and make executable) or use \texttt{make git}
  \begin{itemize}
      \item .git/hooks/post-checkout
      \item .git/hooks/post-commit
      \item .git/hooks/post-merge
  \end{itemize}
  \textbf{Remove watermark}\\
  To disable watermark, remove option \texttt{[mark]} from \textbackslash usepackage[mark]\{gitinfo2\} in \textit{config/commands.tex}.
\end{frame}

\section{Grafiken}
\begin{frame}{Grafiken mit Tikz}
  Grafiken im TikZ-Framework\footnote{\url{http://www.tn-home.de/TUGDD/Stuff/TikZ_final.pdf}} lassen sich mit dem Befehl \textbackslash scaletikzimage\{Datei im Image Verzeichnis\}\{Beschriftung\}\{Label\}\{Skalierungsfaktor\} einbinden. \scaletikzimage{tikz}{TikZ-Grafik}{img:tikz}{0.9}
\end{frame}

\begin{frame}{Grafiken mit moeptikz}
Grafiken mit Netzwerksymbolen können einfach mit moeptikz\footnote{\url{https://github.com/moepinet/moeptikz}} erstellt werden:
\scaletikzimage{network}{Example Network}{img:topo}{0.9}
\end{frame}

\begin{frame}{Grafiken mit Tikz-UML}
UML basierte Grafiken können einfach mit tikz-uml\footnote{\url{https://perso.ensta-paris.fr/~kielbasi/tikzuml/index.php}} erstellt werden:
\scaletikzimage{sequence}{Example Sequence}{img:seq}{0.9}
\end{frame}


\section{Tabellen}
\begin{frame}{Tabellen}
Tabellen lassen sich mit dem Environment\\
\textbackslash begin\{longtable\}[H h t b c]\{Spaltendefinitionen\} ...\\
\qquad\qquad \textbackslash caption\{Tabellenunterschrift\}\\
\qquad\qquad \textbackslash label\{Label\}\\
\textbackslash end\{longtable\}\\
 definieren\footnote{\url{ftp://ftp.dante.de/pub/tex/macros/latex/required/tools/longtable.pdf}}\\
\begin{longtable}[H]{|p{0.2\textwidth}|p{0.2\textwidth}|p{0.2\textwidth}|}
\hline
A&B&C\\
\hline
\caption{Tabelle 1}
\label{tab:tab1}
\end{longtable}
\end{frame}

\section{Code-Ausschnitte}

\begin{frame}[containsverbatim]{Pseudo-Code}
Pseudo-Code Ausschnitte lassen sich mit \textbackslash pseudo\{Name des Algorithmus\}\{Label\}\{Datei im Code-Verzeichnis\} einbinden.
\pseudo{Mittelwert}{lst:mean}{code}
\end{frame}

\begin{frame}[containsverbatim]{Programmiersprachen}
Code Ausschnitte lassen sich einfach mit listings verwenden z.B. in der Umgebung \textbackslash begin\{lstlisting\} ... \textbackslash end\{lstlisting\}, wie in \autoref{lst:python} gezeigt.

\begin{lstlisting}[language=Python, caption={Simple Python program}, label=lst:python]
def my_function():
  print("Hello from a function")
\end{lstlisting}
\end{frame}


%%%%%%%%%%%%%%%%%%%%%%%%%%%%%%%%%%%%%%%%%
%%%%%%%%%% References          %%%%%%%%%%
%%%%%%%%%%%%%%%%%%%%%%%%%%%%%%%%%%%%%%%%%
\section*{}
\begin{frame}[allowframebreaks]{References}
 \def\newblock{\hskip .11em plus .33em minus .07em}
 \scriptsize
 \bibliographystyle{IEEEtran}
 \bibliography{\meta/exampleLiterature/bib}
 \normalsize
\end{frame}

%% Last frame
\frame{
  \vspace{2cm}
  {\huge Questions ?}

  \vspace{20mm}
  \nocite*

  \begin{flushright}
    Marcel Gro\ss mann

    \structure{\footnotesize{\href{mailto:marcel.grossmann@uni-bamberg.de}{marcel.grossmann@uni-bamberg.de}}}
  \end{flushright}
}


\end{document}
